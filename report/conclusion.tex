The result of all three optimizations were basically the same. They each reached a maximum fitness of 55. It is reasonable to conclude that, with the DEFLATE compression used that 55 could be close to the maximum size a 5000 bit output string could compress to. Hence, it is also reasonable to conclude that the GA performed as expected and, for all three cases, found the most complex possible organism. However, since the relative complexity of the best TMs was not significantly higher than the average complexity of majority of the generations, no direct conclusion can be drawn about the universality of the best candidates. A stronger conclusion, though not an absolute one, could be drawn if the maximum complexity was significantly higher than the average. Since this was the case, finding a UTM can neither be rulled out or confirmed. 

In any event, it seems likely that the 5000 character output limit may have unintensionally limited the efficacy of the otimization. Future work will certainly need to greatly increase the limit on the size of the output tape. However, doing so will increase the compuational cost of each fitness evaluation. If the time required for a single fitness evaluation grows, it would become necessary to reconsider the necessity for parallel computation. 

Future research could also include an increasing in the size of alphabet allowed for candidate turing machines. This would require a slight re-design of the encoding method to allow for more than just a single bit to represent the write bit. Making this change might not affect the chances of actually finding a UTM, but it would make it easier to compare the results to known UTMs. There are very few known UTMs with only two symbols in their library, but the list grows signficantly if you allow the libary to increase. Though the comparison would be done during post processing, it would provide a more direct method to evaluate fitness algorithms after the optimziations has completed. 

\clearpage