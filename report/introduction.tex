In 1861 Charles Darwin published his theory of natural selection as a means by which living creatures evolved to best meet the challanges presented to them in their environment. \cite{Origin_of_species} According to Darwin, species undergo changes because certain members of a population posses traits which allow them to reproduce more succefully than others. Those organisms offspring retain the adventeagous traits and hence continue to reproduce more succesfully themselves. It is an interesting thought experiment to concider Darwin's theory in the extreme, where life exists as very simple organisms and progessively evolves into more complex ones eventually resulting in mammals and then primates and then humans. When examined in the extreme, Darwin's theory seems almost impossible. 

We have developed a method to examine this extreme view of Darwin's theory, in an abstract manner, by applying a global optimization process to Turing Machines. Turing Machines, intially concieved as a thought experiemnt by Alan Turing in 1937, "are simple abstract computational devices intended to help investigate the extent and limitations of what can be computed". \cite{SEP_turing} These simple machines can be constructed so that they are capable of computations that range in complexity from essentially useless to practially unlimitted. In their most complex form, Turing Machines are given a special designation: Universal Turing Machines (UTM). A UTM can reproduce the output of any other turing machine (even other UTM's), given the same input. We use employ special capability of a UTM as an anlouge to the complexity that exists in modern living organisms. By using Turing Machines as the subjects of a global optimziation, and carefully construcing the function to evaluate the fitness of each Turing Machine, we will attempt to determine if an optimization can be created where the natural result is a UTM. 

There are many methods of global optimization, but the most fitting method for this research is genetic algorithm optimization. Genetic algorithm optimization was developed by [some guy] during the [some span of years]. The algorithm was designed to mimic the evolutionary process described by Darwin's Theory. [Insert simple explaination of the Genetic Algorithm here]. The key to a successful optimization using the Genetic Algorithm is the selection of the fitness function to evaluate the subject of the optimzation with. Our reserach will focus on employing a series of fitness evaluation methods and investigating which ones will produce the most interesting Turing Machines at the end of the optimization. This search process is intended to present a simple simulation of the more complex process of evolution in nature.
