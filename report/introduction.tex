In 1861 Charles Darwin published his theory of natural selection as a means by which living creatures evolved to best meet the challenges presented to them in their environment. \cite{Origin_of_species} According to Darwin, species undergo changes because certain members of a population posses traits which allow them to reproduce more successfully than others. Those organisms offspring retain the advantageous traits and hence continue to reproduce more successfully themselves. It is an interesting thought experiment to consider Darwin's theory in the extreme, where life exists as very simple organisms and progressively evolves into more complex ones eventually resulting in mammals and then primates and then humans. When examined in the extreme, Darwin's theory seems almost impossible. 

We have developed a method to examine this extreme view of Darwin's theory, in an abstract manner, by applying a global optimization process to Turing Machines (TM). TMs, initially conceived as a thought experiment by Alan Turing in 1937, "are simple abstract computational devices intended to help investigate the extent and limitations of what can be computed". \cite{SEP_turing} These simple machines can be constructed so that they are capable of computations that range in complexity from essentially useless to practically unlimited. In their most complex form, TMs are given a special designation: Universal Turing Machines (UTM). A UTM can reproduce the output of any other TM (even other UTM's), given the same input. We employ special capability of a UTM as an analogue to the complexity that exists in modern living organisms. By using TMs as the subjects of a global optimization, and carefully constructing the function to evaluate the fitness of each TM, an evaluation of the possibility that "complex" TMs can evolve from effectively random "simple" TMs is performed. 

