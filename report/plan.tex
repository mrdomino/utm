In order to complete the project, we will need to construct a genetic
model for Turing Machines. For this, we will need an encoding of Turing
Machines in some serial format; a method of scoring Turing Machines that
favors universality; and a genetic algorithm that fits the problem. In
order to actually see our results, we will also need some way of
visualizing them.

\subsection{Turing Machine Encoding} 

We need to come up with a manner of serially encoding Turing Machines
such that they can be sensibly modified by a genetic
algorithm---essentially, a Turing Machine genome.

\subsection{Turing Machine Scoring} 

We are searching for Universal Turing Machines, but without any prior
knowledge of what a Universal Turing Machine looks like. We must
engineer a way of scoring arbitrary Turing Machines such that those
closer to ``universality'' score higher than those farther away.

We can make the following conjecture: a Turing Machine that is closer to
being universal will have more complex (in the Kolmogorov sense) output
than one that is farther from being universal. Then complexity
approximates universality, and we should give more weight to Turing
Machines that produce complicated-looking output, regardless of what it
is.

\subsection{Genetic Algorithm Selection}
\subsection{Results Visualization}
